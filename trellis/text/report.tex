\documentclass[a4paper,12pt]{article}
\usepackage[russian]{babel}
\usepackage[utf8]{inputenc}
\usepackage{amssymb}
\usepackage{graphicx}
\usepackage{listings}

\newtheorem{theorem}{Теорема}
\newtheorem{lemma}{Лемма}
\newtheorem{statement}{Утверждение}
\newtheorem{definition}{Опр.}

\newenvironment{proof}[1][Доказательство]{\begin{trivlist}
\item[\hskip \labelsep {\bfseries #1}]}{\end{trivlist}}

\newcommand{\qed}{\nobreak \ifvmode \relax \else
      \ifdim\lastskip<1.5em \hskip-\lastskip
      \hskip1.5em plus0em minus0.5em \fi \nobreak
      \vrule height0.75em width0.5em depth0.25em\fi}

\begin{document}
\lstset{language=Java}

\section{Введение}

Цель данной работы – разработка программы, эмулирующей работу скрытого канала
с мультипликативной помехой различного уровня интенсивности, использующей TRELLIS коды 
для помехоустойчивого кодирования передаваемой информации. Варианты TRELLIS кодов 
выбираются из рекомендаций международного союза электросвязи ITU-T V.32, V.32bis. 

\section{Постановка задачи}
Задача разбивается на несколько модулей:

\begin{enumerate}
    \item Реализация алгоритма TRELLIS кодирования по рекоммендациям ITU-T V.32, V.32bis.
    \item Имитация канала связи с мультипликативной ошибкой.
    \item Реализация алгоритма декодирования Витерби.
\end{enumerate}

Заметим, что, поскольку разрабатываемое ПО носит исследовательский, а не прикладной характер,
данный модуль разрабатывался  исходя из необходимости обеспечить максимальное удобство при
отладке, иногда даже в ущерб эффективности в плане производительности.

\section{Описание алгоритмов}

\subsection{Алгоритм TRELLIS кодирования}

\subsection{Имитация канала с ошибкой}

\subsection{Алгоритм Витерби для TRELLIS декодирования}

\section{Описание работы програамы}

\section{Зaключение}

\bibliography{report}

\end{document}
